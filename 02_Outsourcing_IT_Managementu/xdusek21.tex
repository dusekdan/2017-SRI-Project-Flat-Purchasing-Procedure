\documentclass[a4paper,12pt]{article}
\usepackage[left=2.5cm,right=2cm,top=3cm]{geometry}
\usepackage[utf8x]{inputenc}
\usepackage[czech]{babel}
\usepackage[IL2]{fontenc}
\usepackage{eso-pic}
\usepackage{graphicx}
\graphicspath{resources/}

\renewcommand{\baselinestretch}{1.2}

\begin{document}

	% Logo FIT
	\AddToShipoutPictureBG{
		\AtPageUpperLeft{\raisebox{-\height}{\includegraphics[scale=0.50]{resources/fit-logo.pdf}}}
	}
	
	% Turn off numbering
	\pagenumbering{gobble}	

	\setlength{\parindent}{0pt}
	\vspace*{10pt}
	\LARGE \textsc{Projekt č. 2}
	\normalsize

	\vspace*{5pt}
	\textit{Projekt do předmětu SRI - Strategické Řízení Informačních Systémů} \\
	\textit{Téma: Analyzujte a porovnejte 4 základní IT Infrastructure Services management Outsourcing delivery} \\
	\textit{Řešitel: Daniel Dušek (xdusek21)}

	\setlength{\parindent}{15pt}
	\setlength{\parskip}{15pt}
	\renewcommand{\baselinestretch}{1.5}
	\vspace*{15pt}

    V samotném začátku je třeba se zamyslet a zkonstatovat jakým stylem fungují a operují malé firmy a v kontrastu k tomu, jak fungují a operují ty velké. 
    
    Velké firmy, velké zakázky, velké nároky na výnosnost produktu - je potřeba udělat dostatečně velkou zakázku, při které organizace vydělá dostatečné množství finančních prostředků, aby tuto celou firmu uživila a zároveň ještě zvyšovala svůj kapitál. S velkými zakázkami však přicházejí vysoká rizika na možné ztráty při jejich nedodání či při jejich selhání. Čím větší je firma, tím se přirozeně vyskytuje více faktorů, které mohou nakonec snížit jak efektivitu procesu tvorby produktu, tak i samotnou konečnou hodnotu výsledného produktu. Jedno z~rizik, které mohou nastávat jsou samotní zaměstnanci, kteří jako jednotlivci nezodpovídají přímo za výsledný produkt, ale pouze za dokončení jejich dílčího úkolu, kde i zodpovědnost za tento dílčí úkol nemusí být zaměstnancem vnímána jako tak klíčová. Zaměstnance teoreticky jeho chyba může stát \uv{pouze} jeho místo, zatímco firmu to může stát jak velké množství peněz, tak stejně tak dobře i ztrátu dobrého jména. 

    Dále se nabízí potřeba konkurenceschopnosti podniku, která je jistě důležitá jak u malých podniků, tak i u velkých firem. Při naplňování potřeby konkurenceschopnosti firmy často vyvstává potřeba přecházet na nové technologie, popřípadě inovovat. Lze konstatovat, že menší firma bude pravděpodobně \uv{flexibilnější} z pohledu výměny zastarávající technologie za novou, nabrání nových zkušeností, znalostí a schopností, neboť je potřeba zasáhnout pouze malý počet prostředků - ať už lidských nebo infrastrukturních. U větších firem může být přechod na novější technologie nebo rozšíření nových zkušeností, znalostí a schopností o dost náročnější. Nejde však pouze o náročnost zasažení velkého počtu prostředků, ale i~o~rozsáhlejší nastavené procesy, které mohou flexibilnímu pokroku bránit. Tento aspekt je u~větších a velkých firem už značně hůře korigovatelný ve srovnání s malými firmami.

    Oblast informačních technologií se dnes neustále rozšiřuje a posouvá směrem dopředu. K pokroku musí docházet v malých i ve velkých firmách bez rozdílů. A právě zde může outsourcing krásně suplementovat funkci motoru při přesouvání firmy směrem k novějším technologiím a inovacím. Najmutí outsourcingového týmu může pomoci nastartovat změnu, protože je předpoklad, že najmutý outsourcingový tým má již potřebné zkušenosti, znalosti, nástroje a vlastní \uv{know-how} pro oblast, na kterou cílí. Výhodou najmutí outsourcingového týmu může být tedy právě rychlý posun směrem, kterým by se firma jako celek posunout tak rychle nestihla. Za nespornou nevýhodu je zase možné považovat fakt, že tak ale nejsou rozvíjeny vlastní zdroje firmy v novém směru. Tato nevýhoda je zjevně kompenzovatelná školeními, aby zaměstnanci nezůstávali pozadu a neustále se rozvíjeli.

    Podíváme-li se na 4 základní modely a porovnáme-li je vzájemně, uvědomíme si, že pro~firmy různých velikostí jsou vhodné různé modely, nebo, že některé modely jsou pro~některý typ firem vcelku nepoužitelné. 

    Uvažujme případ outsourcování dle \textit{Staff Augmentation} strategie a konkrétní případ menší firmy. Menší softwarová firma má nějakou menší hardwareovou infrastrukturu, která potřebuje čas od času provést údržbu, nainstalovat nutné aktualizace, popřípadě vyměnit zastarávající zařízení. Cílem údržby je se vyhnout kritickým poruchám, které ohrozí denní operace firmy a jde o formu snižování rizika kritické poruchy. Malá firma s menší infrastrukturou očekává méně časté problémy, které by vyžadovaly okamžité řešení a vedení firmy tedy rozhodne, že snižování rizika prováděním pravidelné údržby je pro firmu výhodné řešení. Z~pohledu malé firmy nemá smysl pro provádění pravidelné údržby jednou za čas dedikovat prostředky na rozšíření této dovednosti mezi zaměstnance. V takovém případě tedy malá firma najme někoho externě jako kontraktora, který údržbu provede. 

    Z pohledu velké firmy ovšem může být outsourcování dle \textit{Staff Augmentation} strategie také výhodné, avšak v jiných případech. Má-li obrovská firma několik divizí developerů a~jejich produkt je použit v kritických procesech, kde informační bezpečnost hraje roli, pravděpodobně bude mít firma potřebu zajistit, že jejich produkt bude splňovat náležitosti z~hlediska informační bezpečnosti. Ačkoliv firma bude mít pravděpodobně také oddělení informační bezpečnostni, riziko implementace nezabezpečeného produktu tím nemizí. Riziko, že produkt nebude dostatečně zabezpečný je nejlepší porážet již v zárodku a tedy dostatečnou proškoleností a znalostí základů informační bezpečnosti u všech divizí developerů. Oddělení informační bezpečnosti má svou vlastní agendu a vlastní plány, které musí naplňovat, ve~kterých není počítáno s tím, že zaměstnanci tohoto oddělení budou pravidelně školit stávající i~nové developery v oblasti informační bezpečnosti. V takovém případě může velká společnost benefitovat z najmutí externího kontraktora jehož konkrétním úkolem bude provést sérii školení na vybraná témata z informační bezpečnosti s praktickými demonstracemi. 

    Využití \textit{Out tasking} strategie menší firmou v oblasti softwarového vývoje může být vhodné v případě, že firma má potřebu zvýšit svou konkurenceschopnost například vysokou kvalitou návrhu aplikačního řešení, které dodává společně s jedním z konkrétních typů svých zakázek. Tuto potřebu je možné uspokojit najmutím týmu konzultantů se specializací na návrh software z architektonického hlediska, kteří pomůžou s vytvořením konkurenceschopné architektury pro řešení, které menší firma plánuje dodávat. 

    U větší firmy aplikace \textit{Out tasking} strategie může mít například podobu outsourcování testování části aplikace těsně před vydáním. Uvažujme scénář, kdy existuje 1000 hodin manuálních testovacích scénářů z nichž dobrých 600 hodin vyžaduje pouze povrchovou znalost systému, ale stráví spoustu času. Zbývajících 400 hodin scénářů vyžaduje hlubší znalost a~musí být prováděno v rámci firmy. Outsourcing 600 hodin manuálního testování na specializovanou firmu dodávající kvalitní manuální testery, dle \textit{Out tasking} strategie tu dává smysl, protože velká firma nemusí alokovat tolik jednotek svých zaměstnanců na testování a může je využít k opravování nalezených problémů a tím přispět k celkové vyšší kvalitě vydávaného produktu.

    \textit{Project based} outsourcing pro malou firmu už může být na pováženou, neboť jestliže firma o méně než 50 zaměstnancích zvažuje outsourcing na bázi outsourcování celého projektu, je možné, že se pokouší o realizaci zakázky mimo obor své působnosti. Konkrétním scénářem, který by nesplňoval uvedenou domněnku by mohlo být outsourcování celého testování dodávané aplikace na jinou specializovanou testovací firmu. Je-li měřítko dosahování cílů firmy počet nově naimplementovaných funkčních modulů do aplikace, je pro firmu pravděpodobně \textit{Project based outsourcing} použitelný a nasaditelný.

    Pro velkou firmu \textit{Project based} outsourcing může mít význam při dlouhodobém přetížení zakázkami a neschopností přibírat a efektivně řešit další, než najme a zaškolí nové lidi. Aby si firma udržela majoritní postavení na trhu ve své oblasti, může zvážit outsourcing některých ze zakázek jako celý projekt, aby uměle snížila zátěž a udržela si stávající klienty a zároveň nemusela odmítat ty nové.

    Outsourcing dle \textit{Managed services} strategie pravděpodobně postrádá smysl pro malou firmu do 50 zaměstnanců.

    U velké firmy je outsourcing dle \textit{Managed services} strategie poměrně běžnou praxí. Mnoho velkých softwarových firem však neoutsourcuje oddělení zajišťování kvality a testování, ale spíše oddělení poskytující technickou podporu produktu. Firma může mít například potřebu telefonické podpory produktu v různých časových pásmech a kulturách. Outsourcováním celého projektu poskytování podpory zákazníkům ke konkrétnímu produktu tuto potřebu uspokojí. Výhodou zřejmě je, že takto outsourcovaná divize, se kterou firma dlouhodobě spolupracuje, se postupem času taktéž inovuje a vzdělává a firma, která outsourcovala službu z tohoto taktéž profituje.

    V závěru je tedy možné konstatovat, že existuje-li nutná potřeba outsourcingu, je vhodné pro malou firmu do 50 zaměstnanců uvažovat nad implementací outsourcingu dle \textit{Staff augmentation}, popř. \textit{Out tasking} strategií, přičemž i \textit{Project based outsourcing} strategie může být schůdným řešením, které firmě může být prospěšné, avšak již ne tolik tradiční. Pro velké firmy od 50~000 zaměstnanců je nicméně možno zvolit jakoukoliv implementaci outsourcingu, ovšem v závislosti na potřebách a typech projektů, které outsourcing vyžadují.

\end{document}